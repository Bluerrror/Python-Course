\documentclass{beamer}
\usepackage[utf8]{inputenc}
\usepackage{xcolor}
\usepackage{graphicx}
\usepackage{booktabs}
\usepackage{amsmath}
\usepackage{multirow}
\usepackage{fontawesome}
\usepackage{tikz}
\usepackage{caption}
\usepackage{hyperref}
\usepackage{listings}
\usepackage{tcolorbox}

% Define custom colors
\definecolor{darkblue}{RGB}{0, 51, 102}
\definecolor{lightblue}{RGB}{173, 216, 230}
\definecolor{orange}{RGB}{255, 165, 0}
\definecolor{darkgray}{RGB}{50, 50, 50}

% Custom header and footer
\setbeamercolor{item}{fg=darkblue}
\setbeamercolor{block title}{bg=darkblue, fg=white}
\setbeamercolor{block body}{bg=lightblue, fg=black}

% Font settings
\usefonttheme{professionalfonts} % Use professional fonts
\usepackage{lmodern} % Latin Modern font

% Code styling
\lstset{
    basicstyle=\ttfamily\footnotesize,
    backgroundcolor=\color{lightblue},
    frame=single,
    rulecolor=\color{darkblue},
    keywordstyle=\color{orange}\bfseries,
    commentstyle=\color{darkgray},
    stringstyle=\color{darkgray},
    numbers=left,
    numberstyle=\tiny\color{darkgray},
    stepnumber=1,
    numbersep=5pt,
    tabsize=2,
    breaklines=true
}

\title{\textbf{Getting Started with Python: Syntax, Functions, and Classes}}
\author{\textbf{Shahab A. Shojaeezadeh} \\ \faEnvelope \, shahab2710@gmail.com \\ \faGithub \, github.com/shahab271069}
\date{November 04, 2024}


% Course name
\newcommand{\courseName}{Introduction to Programming with Python}

% Logo path definitions
\newcommand{\logoOne}{boku.png} % Change this to your logo path
\newcommand{\logoTwo}{uni-kassel.png} % Change this to your logo path

% Custom footline with course name on left and page number on right
\setbeamertemplate{footline}{
    \begin{beamercolorbox}[wd=\textwidth,ht=2ex,dp=0ex]{}
        \textcolor{darkblue}{\courseName} % Course name on the left
        \hspace{5pt} % Space below logos
        \vspace{2pt} % Space below logos
        \hfill
        \textcolor{darkblue}{\insertframenumber} / \inserttotalframenumber % Page number on the right
        \hspace{2pt} % Space below logos
        \vspace{2pt} % Space below logos
    \end{beamercolorbox}
}

\begin{document}
% Title page with logos
\begin{frame}[plain] % Use plain option to avoid default header/footer
    \begin{center}
        \includegraphics[height=1.5cm]{\logoOne} % Left logo
        \hfill
        \includegraphics[height=1.5cm]{\logoTwo} % Right logo
        %\vspace{5pt} % Space below logos
        \titlepage
    \end{center}
\end{frame}

% Remove header for other frames
\setbeamertemplate{headline}{}

% Agenda
\begin{frame}
    \frametitle{Agenda}
    \begin{itemize}
        \item Importing Packages
        \item Using `os`
        \item Using `glob`
        \item Using `numpy`
        \item Using `scipy`
        \item Using `matplotlib`
        \item Using `pandas`
        \item Summary and Questions
    \end{itemize}
\end{frame}

% Importing Packages
\begin{frame}
    \frametitle{Importing Packages}
    \begin{itemize}
        \item Use the \texttt{import} statement
        \item Example:
        \begin{tcolorbox}[colback=lightblue, colframe=darkblue, title=Import Example]
            \lstinline|import os| \\
            \lstinline|import pandas as pd| \\
            \lstinline|import glob| \\
            \lstinline|import matplotlib.pyplot as plt| \\
            \lstinline|import numpy as np| \\
            \lstinline|from scipy import stats|
        \end{tcolorbox}
    \end{itemize}
\end{frame}

% Using `os`
\begin{frame}
    \frametitle{Using `os` Package}
    \begin{itemize}
        \item Interact with the operating system
        \item Example: Listing files in a directory
        \begin{tcolorbox}[colback=lightblue, colframe=darkblue, title=List Files]
            \lstinline|import os| \\
            \lstinline|files = os.listdir(".")| \\
            \lstinline|print(files)|
        \end{tcolorbox}
    \end{itemize}
\end{frame}

% Using `os` - Check File Existence
\begin{frame}
    \frametitle{Using `os` - Check File Existence}
    \begin{itemize}
        \item Check if a file exists
        \item Example:
        \begin{tcolorbox}[colback=lightblue, colframe=darkblue, title=Check File]
            \lstinline|if os.path.exists("file.txt"):| \\
            \hspace{10pt} \lstinline|print("File exists")|
        \end{tcolorbox}
    \end{itemize}
\end{frame}

% Using `os` - Create a Directory
\begin{frame}
    \frametitle{Using `os` - Create a Directory}
    \begin{itemize}
        \item Create a new directory
        \item Example:
        \begin{tcolorbox}[colback=lightblue, colframe=darkblue, title=Create Directory]
            \lstinline|os.mkdir("new_folder")| \\
            \lstinline|print("Directory created")|
        \end{tcolorbox}
    \end{itemize}
\end{frame}

% Using `os` - Remove a File
\begin{frame}
    \frametitle{Using `os` - Remove a File}
    \begin{itemize}
        \item Delete a file
        \item Example:
        \begin{tcolorbox}[colback=lightblue, colframe=darkblue, title=Remove File]
            \lstinline|os.remove("file.txt")| \\
            \lstinline|print("File removed")|
        \end{tcolorbox}
    \end{itemize}
\end{frame}

% Using `os` - Get Current Working Directory
\begin{frame}
    \frametitle{Using `os` - Get Current Working Directory}
    \begin{itemize}
        \item Get the current working directory
        \item Example:
        \begin{tcolorbox}[colback=lightblue, colframe=darkblue, title=Current Directory]
            \lstinline|cwd = os.getcwd()| \\
            \lstinline|print(cwd)|
        \end{tcolorbox}
    \end{itemize}
\end{frame}

% Using `glob`
\begin{frame}
    \frametitle{Using `glob` Package}
    \begin{itemize}
        \item Find all the pathnames matching a specified pattern
        \item Example: List all .txt files
        \begin{tcolorbox}[colback=lightblue, colframe=darkblue, title=Glob Example]
            \lstinline|import glob| \\
            \lstinline|files = glob.glob("*.txt")| \\
            \lstinline|print(files)|
        \end{tcolorbox}
    \end{itemize}
\end{frame}

% Using `glob` - Recursively Finding Files
\begin{frame}
    \frametitle{Using `glob` - Recursively Finding Files}
    \begin{itemize}
        \item Search for files in subdirectories
        \item Example:
        \begin{tcolorbox}[colback=lightblue, colframe=darkblue, title=Recursive Search]
            \lstinline|files = glob.glob("**/*.txt", recursive=True)| \\
            \lstinline|print(files)|
        \end{tcolorbox}
    \end{itemize}
\end{frame}

% Using `numpy`
\begin{frame}
    \frametitle{Using `numpy` Package}
    \begin{itemize}
        \item Fundamental package for numerical computing
        \item Example: Creating an array
        \begin{tcolorbox}[colback=lightblue, colframe=darkblue, title=Create Array]
            \lstinline|import numpy as np| \\
            \lstinline|arr = np.array([1, 2, 3])| \\
            \lstinline|print(arr)|
        \end{tcolorbox}
    \end{itemize}
\end{frame}

% Using `numpy` - Array Operations
\begin{frame}
    \frametitle{Using `numpy` - Array Operations}
    \begin{itemize}
        \item Perform operations on arrays
        \item Example: Element-wise addition
        \begin{tcolorbox}[colback=lightblue, colframe=darkblue, title=Array Addition]
            \lstinline|arr1 = np.array([1, 2, 3])| \\
            \lstinline|arr2 = np.array([4, 5, 6])| \\
            \lstinline|result = arr1 + arr2| \\
            \lstinline|print(result)|
        \end{tcolorbox}
    \end{itemize}
\end{frame}

% Using `numpy` - Statistical Functions
\begin{frame}
    \frametitle{Using `numpy` - Statistical Functions}
    \begin{itemize}
        \item Perform statistical calculations
        \item Example: Mean and Standard Deviation
        \begin{tcolorbox}[colback=lightblue, colframe=darkblue, title=Statistics]
            \lstinline|mean = np.mean(arr)| \\
            \lstinline|std_dev = np.std(arr)| \\
            \lstinline|print(mean, std_dev)|
        \end{tcolorbox}
    \end{itemize}
\end{frame}

% Using `numpy` - Reshape Array
\begin{frame}
    \frametitle{Using `numpy` - Reshape Array}
    \begin{itemize}
        \item Change the shape of an array
        \item Example:
        \begin{tcolorbox}[colback=lightblue, colframe=darkblue, title=Reshape Array]
            \lstinline|reshaped = arr.reshape(3, 1)| \\
            \lstinline|print(reshaped)|
        \end{tcolorbox}
    \end{itemize}
\end{frame}

% Using `scipy`
\begin{frame}
    \frametitle{Using `scipy` Package}
    \begin{itemize}
        \item Library for scientific computing
        \item Example: Statistical tests
        \begin{tcolorbox}[colback=lightblue, colframe=darkblue, title=Statistical Tests]
            \lstinline|from scipy import stats| \\
            \lstinline|t_statistic, p_value = stats.ttest_ind(sample1, sample2)| \\
            \lstinline|print(t_statistic, p_value)|
        \end{tcolorbox}
    \end{itemize}
\end{frame}

% Using `scipy` - Optimization
\begin{frame}
    \frametitle{Using `scipy` - Optimization}
    \begin{itemize}
        \item Solve optimization problems
        \item Example:
        \begin{tcolorbox}[colback=lightblue, colframe=darkblue, title=Optimization Example]
            \lstinline|from scipy.optimize import minimize| \\
            \lstinline|result = minimize(fun, x0)| \\
            \lstinline|print(result)| 
        \end{tcolorbox}
    \end{itemize}
\end{frame}

% Using `scipy` - Interpolation
\begin{frame}
    \frametitle{Using `scipy` - Interpolation}
    \begin{itemize}
        \item Perform interpolation on data
        \item Example:
        \begin{tcolorbox}[colback=lightblue, colframe=darkblue, title=Interpolation]
            \lstinline|from scipy.interpolate import interp1d| \\
            \lstinline|f = interp1d(x, y)| \\
            \lstinline|new_y = f(new_x)| 
        \end{tcolorbox}
    \end{itemize}
\end{frame}

% Using `matplotlib`
\begin{frame}
    \frametitle{Using `matplotlib` Package}
    \begin{itemize}
        \item Create static, animated, and interactive visualizations
        \item Example: Simple Line Plot
        \begin{tcolorbox}[colback=lightblue, colframe=darkblue, title=Line Plot]
            \lstinline|import matplotlib.pyplot as plt| \\
            \lstinline|plt.plot([1, 2, 3], [4, 5, 6])| \\
            \lstinline|plt.show()|
        \end{tcolorbox}
    \end{itemize}
\end{frame}

% Using `matplotlib` - Scatter Plot
\begin{frame}
    \frametitle{Using `matplotlib` - Scatter Plot}
    \begin{itemize}
        \item Create scatter plots
        \item Example:
        \begin{tcolorbox}[colback=lightblue, colframe=darkblue, title=Scatter Plot]
            \lstinline|plt.scatter(x, y)| \\
            \lstinline|plt.title("Scatter Plot")| \\
            \lstinline|plt.xlabel("X-axis")| \\
            \lstinline|plt.ylabel("Y-axis")| \\
            \lstinline|plt.show()|
        \end{tcolorbox}
    \end{itemize}
\end{frame}

% Using `matplotlib` - Bar Plot
\begin{frame}
    \frametitle{Using `matplotlib` - Bar Plot}
    \begin{itemize}
        \item Create bar charts
        \item Example:
        \begin{tcolorbox}[colback=lightblue, colframe=darkblue, title=Bar Plot]
            \lstinline|plt.bar(["A", "B", "C"], [1, 2, 3])| \\
            \lstinline|plt.title("Bar Plot")| \\
            \lstinline|plt.show()|
        \end{tcolorbox}
    \end{itemize}
\end{frame}

% Using `matplotlib` - Histogram
\begin{frame}
    \frametitle{Using `matplotlib` - Histogram}
    \begin{itemize}
        \item Create histograms
        \item Example:
        \begin{tcolorbox}[colback=lightblue, colframe=darkblue, title=Histogram]
            \lstinline|plt.hist(data)| \\
            \lstinline|plt.title("Histogram")| \\
            \lstinline|plt.show()|
        \end{tcolorbox}
    \end{itemize}
\end{frame}


% Using `pandas`
\begin{frame}
    \frametitle{Using `pandas` Package}
    \begin{itemize}
        \item Data manipulation and analysis
        \item Example: Reading a CSV file
        \begin{tcolorbox}[colback=lightblue, colframe=darkblue, title=Read CSV]
            \lstinline|import pandas as pd| \\
            \lstinline|data = pd.read_csv("data.csv")| \\
            \lstinline|print(data.head())|
        \end{tcolorbox}
    \end{itemize}
\end{frame}

% Using `pandas` - DataFrame Operations
\begin{frame}
    \frametitle{Using `pandas` - DataFrame Operations}
    \begin{itemize}
        \item Perform operations on DataFrames
        \item Example: Filtering Data
        \begin{tcolorbox}[colback=lightblue, colframe=darkblue, title=Filter DataFrame]
            \lstinline|filtered = data[data["column"] > 10]| \\
            \lstinline|print(filtered)|
        \end{tcolorbox}
    \end{itemize}
\end{frame}

% Using `pandas` - Group By
\begin{frame}
    \frametitle{Using `pandas` - Group By}
    \begin{itemize}
        \item Group data by a specific column
        \item Example:
        \begin{tcolorbox}[colback=lightblue, colframe=darkblue, title=Group By]
            \lstinline|grouped = data.groupby("column").mean()| \\
            \lstinline|print(grouped)|
        \end{tcolorbox}
    \end{itemize}
\end{frame}

% Using `pandas` - Merging DataFrames
\begin{frame}
    \frametitle{Using `pandas` - Merging DataFrames}
    \begin{itemize}
        \item Combine multiple DataFrames
        \item Example:
        \begin{tcolorbox}[colback=lightblue, colframe=darkblue, title=Merging DataFrames]
            \lstinline|merged = pd.merge(df1, df2, on="key")| \\
            \lstinline|print(merged)|
        \end{tcolorbox}
    \end{itemize}
\end{frame}

% Using `pandas` - Saving DataFrame
\begin{frame}
    \frametitle{Using `pandas` - Saving DataFrame}
    \begin{itemize}
        \item Save DataFrame to CSV
        \item Example:
        \begin{tcolorbox}[colback=lightblue, colframe=darkblue, title=Save DataFrame]
            \lstinline|data.to_csv("output.csv", index=False)|
        \end{tcolorbox}
    \end{itemize}
\end{frame}

% Summary
\begin{frame}
    \frametitle{Summary}
    \begin{itemize}
        \item Covered essential Python packages
        \item Demonstrated code examples for each package
        \item Discussed practical applications
    \end{itemize}
\end{frame}

% Questions
\begin{frame}
    \frametitle{Questions?}
    \begin{center}
        Thank you for your attention! \\
        Any questions?
    \end{center}
\end{frame}

\end{document}
