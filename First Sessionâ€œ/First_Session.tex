\documentclass{beamer}
\usepackage[utf8]{inputenc}
\usepackage{xcolor}
\usepackage{graphicx}
\usepackage{booktabs}
\usepackage{amsmath}
\usepackage{multirow}
\usepackage{fontawesome}
\usepackage{tikz}
\usepackage{caption}
\usepackage{hyperref}
\usepackage{listings}
\usepackage{tcolorbox}

% Define custom colors
\definecolor{darkblue}{RGB}{0, 51, 102}
\definecolor{lightblue}{RGB}{173, 216, 230}
\definecolor{orange}{RGB}{255, 165, 0}
\definecolor{darkgray}{RGB}{50, 50, 50}

% Custom header and footer
\setbeamertemplate{footline}[frame number]
\setbeamercolor{item}{fg=darkblue}
\setbeamercolor{block title}{bg=darkblue, fg=white}
\setbeamercolor{block body}{bg=lightblue, fg=black}

% Font settings
\usefonttheme{professionalfonts} % Use professional fonts
\usepackage{lmodern} % Latin Modern font

% Code styling
\lstset{
    basicstyle=\ttfamily\footnotesize,
    backgroundcolor=\color{lightblue},
    frame=single,
    rulecolor=\color{darkblue},
    keywordstyle=\color{orange}\bfseries,
    commentstyle=\color{darkgray},
    stringstyle=\color{darkgray},
    numbers=left,
    numberstyle=\tiny\color{darkgray},
    stepnumber=1,
    numbersep=5pt,
    tabsize=2,
    breaklines=true
}

\title{\textbf{Getting Started with Python}}
\author{\textbf{Shahab A. Shojaeezadeh} \\ \faEnvelope \, shahab2710@gmail.com \\ \faGithub \, github.com/shahab271069}
\date{October 14, 2024}

\begin{document}

\frame{\titlepage}

\begin{frame}
    \frametitle{Agenda}
    \begin{itemize}
        \item Introduction to Python
        \item Installing Python
        \item Setting up a Virtual Environment
        \item Using Conda
        \item Installing JupyterLab/Notebook
        \item Installing IDEs
        \item Writing Your First Python Program
        \item Using GitHub
        \item Resources for Learning
    \end{itemize}
\end{frame}

\begin{frame}
    \frametitle{What is Python?}
    \begin{itemize}
        \item High-level programming language
        \item Interpreted and easy to learn
        \item Widely used in web development, data science, automation, etc.
    \end{itemize}
\end{frame}

\begin{frame}
    \frametitle{Why Use Python?}
    \begin{itemize}
        \item Extensive libraries and frameworks
        \item Large community support
        \item Versatile and powerful
    \end{itemize}
\end{frame}

\begin{frame}
    \frametitle{System Requirements}
    \begin{itemize}
        \item Windows, macOS, or Linux
        \item Internet connection for downloads
        \item Basic command line knowledge
    \end{itemize}
\end{frame}

\begin{frame}
    \frametitle{Downloading Python}
    \begin{itemize}
        \item Visit the official Python website: \texttt{python.org}
        \item Click on the \textbf{"Downloads"} section
        \item Choose the appropriate version for your OS
    \end{itemize}
\end{frame}

\begin{frame}
    \frametitle{Installing Python on Windows}
    \begin{itemize}
        \item Run the installer
        \item Check \texttt{"Add Python to PATH"}
        \item Follow the installation prompts
        \item Install necessary packages:
        \begin{tcolorbox}[colback=lightblue, colframe=darkblue, title=Install Packages]
            \lstinline|pip install numpy pandas matplotlib|
        \end{tcolorbox}
    \end{itemize}
\end{frame}

\begin{frame}
    \frametitle{Installing Python on macOS}
    \begin{itemize}
        \item Open the downloaded \texttt{.pkg} file
        \item Follow the installation instructions
        \item Verify installation with:
        \begin{tcolorbox}[colback=lightblue, colframe=darkblue, title=Verify Installation]
            \lstinline|python3 --version|
        \end{tcolorbox}
    \end{itemize}
\end{frame}

\begin{frame}
    \frametitle{Installing Python on Linux}
    \begin{itemize}
        \item Use the package manager (e.g., APT for Ubuntu)
        \item Command:
        \begin{tcolorbox}[colback=lightblue, colframe=darkblue, title=Install Python]
            \lstinline|sudo apt install python3 python3-pip|
        \end{tcolorbox}
        \item Verify installation:
        \begin{tcolorbox}[colback=lightblue, colframe=darkblue, title=Verify Installation]
            \lstinline|python3 --version|
        \end{tcolorbox}
        \item Install necessary packages:
        \begin{tcolorbox}[colback=lightblue, colframe=darkblue, title=Install Packages]
            \lstinline|pip3 install numpy pandas matplotlib|
        \end{tcolorbox}
    \end{itemize}
\end{frame}

\begin{frame}
    \frametitle{Using Conda}
    \begin{itemize}
        \item Install Anaconda or Miniconda from \texttt{anaconda.com}
        \item Create a new environment:
        \begin{tcolorbox}[colback=lightblue, colframe=darkblue, title=Create Environment]
            \lstinline|conda create --name myenv python=3.x|
        \end{tcolorbox}
        \item Activate the environment:
        \begin{tcolorbox}[colback=lightblue, colframe=darkblue, title=Activate Environment]
            \lstinline|conda activate myenv|
        \end{tcolorbox}
        \item Install packages:
        \begin{tcolorbox}[colback=lightblue, colframe=darkblue, title=Install Packages]
            \lstinline|conda install numpy pandas matplotlib|
        \end{tcolorbox}
    \end{itemize}
\end{frame}

\begin{frame}
    \frametitle{Installing JupyterLab/Notebook}
    \begin{itemize}
        \item Install via pip:
        \begin{tcolorbox}[colback=lightblue, colframe=darkblue, title=Install Jupyter Notebook]
            \lstinline|pip install notebook| \text{ (for Jupyter Notebook)}
        \end{tcolorbox}
        \begin{tcolorbox}[colback=lightblue, colframe=darkblue, title=Install JupyterLab]
            \lstinline|pip install jupyterlab| \text{ (for JupyterLab)}
        \end{tcolorbox}
        
    \end{itemize}
\end{frame}


\begin{frame}
    \frametitle{Installing JupyterLab/Notebook}
    \begin{itemize}
    \item Install via Conda:
            \begin{tcolorbox}[colback=lightblue, colframe=darkblue, title=Install Jupyter Notebook]
                \lstinline|conda install -c conda-forge notebook| \text{ (for Jupyter Notebook)}
            \end{tcolorbox}
            \begin{tcolorbox}[colback=lightblue, colframe=darkblue, title=Install JupyterLab]
                \lstinline|conda install -c conda-forge jupyterlab| \text{ (for JupyterLab)}
            \end{tcolorbox}
            \item Launch Jupyter:
            \begin{tcolorbox}[colback=lightblue, colframe=darkblue, title=Launch Jupyter]
                \lstinline|jupyter notebook| \text{ or } \lstinline|jupyter lab|
            \end{tcolorbox}
    \end{itemize}
\end{frame}

\begin{frame}
    \frametitle{Setting Up a Virtual Environment}
    \begin{itemize}
        \item Use virtual environments to manage dependencies
        \item Command:
        \begin{tcolorbox}[colback=lightblue, colframe=darkblue, title=Create Virtual Environment]
            \lstinline|python -m venv myenv|
        \end{tcolorbox}
        \item Activate the environment:
        \begin{itemize}
            \item Windows:
            \begin{tcolorbox}[colback=lightblue, colframe=darkblue, title=Activate Windows]
                \lstinline|myenv\textbackslash Scripts\textbackslash activate|
            \end{tcolorbox}
            \item macOS/Linux:
            \begin{tcolorbox}[colback=lightblue, colframe=darkblue, title=Activate macOS/Linux]
                \lstinline|source myenv/bin/activate|
            \end{tcolorbox}
        \end{itemize}
    \end{itemize}
\end{frame}

\begin{frame}
    \frametitle{Installing Packages with pip}
    \begin{itemize}
        \item \texttt{pip} is the package installer for Python
        \item Command to install a package:
        \begin{tcolorbox}[colback=lightblue, colframe=darkblue, title=Install Package]
            \lstinline|pip install package_name|
        \end{tcolorbox}
        \item Example:
        \begin{tcolorbox}[colback=lightblue, colframe=darkblue, title=Install Numpy]
            \lstinline|pip install numpy|
        \end{tcolorbox}
    \end{itemize}
\end{frame}

\begin{frame}
    \frametitle{Choosing an IDE}
    \begin{itemize}
        \item Popular IDEs:
        \begin{itemize}
            \item \textbf{PyCharm}
            \item \textbf{VSCode}
            \item \textbf{Jupyter Notebook/Lab}
        \end{itemize}
    \end{itemize}
\end{frame}

\begin{frame}
    \frametitle{Installing PyCharm}
    \begin{itemize}
        \item Download from \texttt{jetbrains.com/pycharm}
        \item Choose Community edition for free version
        \item Follow installation steps
    \end{itemize}
\end{frame}

\begin{frame}
    \frametitle{Installing VSCode}
    \begin{itemize}
        \item Download from \texttt{code.visualstudio.com}
        \item Follow installation instructions
        \item Install Python extension from the marketplace
    \end{itemize}
\end{frame}

\begin{frame}
    \frametitle{Writing Your First Python Program}
    \begin{itemize}
        \item Open your IDE
        \item Create a new file: \texttt{hello.py}
        \item Write:
        \begin{tcolorbox}[colback=lightblue, colframe=darkblue, title=Hello World Program]
            \lstinline|print("Hello, World!")|
        \end{tcolorbox}
    \end{itemize}
\end{frame}

\begin{frame}
    \frametitle{Running Your Python Program}
    \begin{itemize}
        \item Open terminal/command prompt
        \item Navigate to the file location
        \item Run:
        \begin{tcolorbox}[colback=lightblue, colframe=darkblue, title=Run Program]
            \lstinline|python hello.py|
        \end{tcolorbox}
    \end{itemize}
\end{frame}

\begin{frame}
    \frametitle{Using GitHub}
    \begin{itemize}
        \item Create a GitHub account: \texttt{github.com}
        \item Install Git:
        \begin{tcolorbox}[colback=lightblue, colframe=darkblue, title=Install Git]
            \lstinline|sudo apt install git| \text{ (Linux)}
        \end{tcolorbox}
        \begin{tcolorbox}[colback=lightblue, colframe=darkblue, title=Clone Repository]
            \lstinline|git clone <repository_url>|
        \end{tcolorbox}
        
        \end{itemize}

\end{frame}


\begin{frame}
    \frametitle{Common Issues}
    \begin{itemize}
        \item Installation errors
        \item PATH issues
        \item Package conflicts
    \end{itemize}
\end{frame}

\begin{frame}
    \frametitle{Helpful Resources}
    \begin{itemize}
        \item Official Python Documentation: \texttt{docs.python.org}
        \item Online tutorials: Codecademy, Coursera
        \item Community forums: Stack Overflow, Reddit
    \end{itemize}
\end{frame}

\begin{frame}
    \frametitle{Summary}
    \begin{itemize}
        \item Python is easy to install and use
        \item Conda provides a robust package management system
        \item JupyterLab/Notebook is great for interactive coding
        \item GitHub enhances collaboration in coding projects
        \item Many resources available to help you learn
    \end{itemize}
\end{frame}

\begin{frame}
    \frametitle{Questions?}
    \begin{center}
        Thank you! Any questions?
    \end{center}
\end{frame}

\end{document}
